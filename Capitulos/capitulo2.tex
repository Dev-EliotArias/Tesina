\newpage
\section{Capitulo II}
\section*{Plan del proyecto de mejora}
%Introduccion del cap 2
En este capítulo, se aborda el análisis y la planificación de las acciones para resolver los problemas identificados en Automotriz PELAYO S.A.C. Se inicia con la identificación detallada de los problemas técnicos utilizando diversas herramientas como la lluvia de ideas y la matriz de priorización. Luego, se establecen los objetivos del proyecto, se revisan antecedentes relevantes y se justifica la necesidad de llevar a cabo el proyecto de mejora. Este capítulo sienta las bases para el desarrollo de acciones concretas que mejorarán la eficiencia operativa y la satisfacción del cliente en la empresa.

\subsection{2.1 Identificación del problema técnico en la empresa}
%Introduccion de la subsection 2.1
En este punto, se llevará a cabo un análisis detallado de los problemas técnicos que afectan el desempeño y la eficiencia operativa de la empresa. Se utilizarán diversas herramientas y técnicas, como la lluvia de ideas, el diagrama de afinidades y la matriz de priorización, para identificar y priorizar estos problemas de manera sistemática.\\
Se examinarán áreas clave de la empresa, como la disponibilidad de recursos humanos, la infraestructura y el equipamiento, así como la gestión de la información. Cada problema técnico identificado será evaluado en función de su frecuencia, importancia y factibilidad, permitiendo una comprensión más profunda de su impacto y urgencia.

\subsubsection*{2.1.1 Trabajo en Equipo}
Para abordar el problema técnico identificado en la empresa, se aplicará la técnica de trabajo en equipo. Esta estrategia implica la participación activa de todos los miembros del personal,
el objetivo es aprovechar la experiencia y el conocimiento de cada uno de ellos para comprender mejor el problema desde diferentes perspectivas.
\subsubsection*{2.1.2 Lluvia de Ideas}
Se empleo la técnica de lluvia de ideas para poder identificar el problema técnico de la empresa. Esta metodología, tiene como objetivo que los participantes aporten sobre los problemas que tienen al realizar su trabajo diario, esta fue la lista de problemas que cada participante indico que tiene en la realización de su trabajo.
\begin{enumerate}
    \item Falta de disponibilidad de piezas y componentes:
    \begin{itemize}
        \item Problema: No se dispone de las piezas y componentes necesarios para realizar las reparaciones de manera oportuna.
        \item Impacto: Aumento en el tiempo de espera para los clientes, disminución de la satisfacción del cliente, y retrasos en la entrega de servicios.
    \end{itemize}    

    \item Deficiente sistema centralizado de registro:
    \begin{itemize}
        \item Problema: La información de los clientes y de los servicios ofrecidos no está bien centralizada, lo que dificulta su acceso y gestión.
        \item Impacto: Ineficiencia en la búsqueda de información, duplicidad de registros y dificultades para ofrecer un servicio al cliente coherente y eficiente.
    \end{itemize}
    
    \item Falta de personal capacitado para tareas específicas:
    \begin{itemize}
        \item Problema: El taller carece de personal con la formación y habilidades necesarias para realizar tareas especializadas.
        \item Impacto: Menor calidad en los servicios ofrecidos, aumento en los errores y trabajos repetidos, y dependencia de personal externo o de otros talleres.
    \end{itemize}
    
    \item Exceso de tareas en cola para el personal:
    \begin{itemize}
        \item Problema: El personal del taller tiene una gran cantidad de tareas acumuladas, lo que dificulta la gestión eficiente del tiempo y los recursos.
        \item Impacto: Aumento en los tiempos de espera para los servicios, incremento en la carga de trabajo del personal y potencial disminución de la calidad del trabajo debido a la sobrecarga.
    \end{itemize}

    \item Deficiencia en la gestión de tiempo de los operarios:
    \begin{itemize}
        \item Problema: La falta de una adecuada planificación y organización del tiempo de los operarios lleva a una gestión ineficiente de las tareas.
        \item Impacto: Menor productividad, retrasos en la finalización de trabajos y posible pérdida de clientes debido a la falta de puntualidad en la entrega de servicios.
    \end{itemize}

    \item Deficiencias en la infraestructura y equipamiento del taller:
    \begin{itemize}
        \item Problema: La infraestructura y el equipamiento del taller son inadecuados o insuficientes para satisfacer la demanda de servicios.
        \item Impacto: Aumento en los tiempos de reparación, limitaciones en la capacidad de servicio y posibles inconvenientes en la calidad del trabajo realizado.
    \end{itemize}

    \item Retrasos en la entrega de servicios debido a carga de trabajo excesiva:
    \begin{itemize}
        \item Problema: La alta demanda de servicios sobrecarga la capacidad del taller, resultando en retrasos.
        \item Impacto: Insatisfacción del cliente, pérdida de clientes potenciales y estrés laboral para los empleados.
    \end{itemize}

    \item Dificultades para gestionar la demanda en períodos de alta actividad:
    \begin{itemize}
        \item Problema: El taller enfrenta dificultades para manejar la alta demanda de servicios durante las temporadas pico, como vacaciones o épocas de mal tiempo.
        \item Impacto: Incremento en los tiempos de espera, baja calidad en el servicio debido a la presión y potencial pérdida de clientes.
    \end{itemize}
    
    \item Retrasos en la facturación:
    \begin{itemize}
        \item Problema: El proceso manual de facturación puede ser lento y propenso a errores.
        \item Impacto: Demoras en la emisión de facturas, insatisfacción de los clientes y posibles problemas de flujo de caja debido a la tardanza en la recepción de pagos.
    \end{itemize}    
    
    \item Falta de mantenimiento regular de las herramientas y equipos:
    \begin{itemize}
        \item Problema: Las herramientas y equipos del taller no reciben mantenimiento adecuado y regular, lo que puede llevar a fallos y a la necesidad de reparaciones frecuentes.
        \item Impacto: Retrasos en la realización de servicios, incremento en los costos operativos y posible disminución en la calidad del trabajo debido a herramientas defectuosas.    
    \end{itemize}    
\end{enumerate}

\subsubsection*{2.1.3 Diagrama de Afinidades}
En este punto se diseñará el diagrama de afinidades en base a la lluvia de ideas generada en el punto anterior. De esta forma, se agruparán los elementos relacionados en categorías clave, facilitando la identificación de soluciones específicas para cada problema. Las categorias elegidas son estas:%El siguiente diagrama divide las ideas, en tres conceptos clave, Registro y Gestión de Datos, gestión de Recursos Humanos e Infraestructura y Equipamiento. De esta forma se facilitará la identificación de soluciones efectivas para cada categoría de problemas.
%tabla 1 para referenciar
\begin{itemize}
    \item Registro y Gestión de Datos: Incluye problemas relacionados con la falta de disponibilidad de piezas y componentes necesarios para las reparaciones, retrasos en la facturación debido a procesos manuales y la falta de un sistema centralizado para el registro de información.
    \item Gestión de Recursos Humanos: Abarca la falta de personal capacitado, la sobrecarga de tareas para el personal y la deficiencia en la gestión del tiempo de los operarios, lo que afecta la calidad del servicio y la eficiencia del taller.
    \item Infraestructura y Equipamiento: Se refiere a las deficiencias en la infraestructura del taller y en el equipamiento, la dificultad para gestionar la alta demanda de servicios y la falta de mantenimiento regular de herramientas y equipos, lo que puede aumentar los costos operativos y provocar retrasos en la entrega de servicios.
\end{itemize}


\definecolor{ChetwodeBlue}{rgb}{0.556,0.662,0.858}
\begin{table}[H]
\caption[Diagrama de afinidades]{Diagrama de afinidades}
\label{tab:afinidides}
\centering
\begin{threeparttable}
\begin{tblr}{
  row{1} = {ChetwodeBlue},
  cell{2}{1} = {r=3}{},
  cell{5}{1} = {r=3}{},
  cell{8}{1} = {r=4}{},
  vlines,
  hline{1-2,5,8,12} = {-}{},
  hline{3-4,6-7,9-11} = {2}{},
  column{1} = {4cm},
  column{2} = {11cm}
}
\textbf{Categoría} & \textbf{Problema}\\
Registro y Gestión de Datos & Dificultades para gestionar la demanda en períodos de alta actividad.\\
 & Retrasos en la facturación debido al proceso
  manual y propenso a errores.\\
 & Deficiente sistema centralizado de registro.\\
Gestión de Recursos Humanos & Falta
  de personal capacitado para tareas específicas.\\
 & Exceso de tareas en cola para el personal.\\
 & Deficiencia
  en la gestión de tiempo de los operarios.\\
Infraestructura y Equipamiento & Deficiencias en
  la infraestructura y equipamiento del taller.\\
 & Retrasos en la entrega de servicios debido a carga de trabajo excesiva\\
 & Falta de disponibilidad de piezas y componentes necesarios para realizar las reparaciones de manera oportuna.\\
 & Retrasos en la entrega de servicios.
\end{tblr}
\begin{tablenotes}
    %\vspace{0.1cm}
    %\footnotesize
    %\item[1] Nota al pie 1: Esta es una descripción detallada de la nota al pie 1.
    %\item[2] Nota al pie 2: Esta es una descripción detallada de la nota al pie 2.
\end{tablenotes}
\end{threeparttable}
\end{table}

\subsubsection*{2.1.4 Matriz de priorización}
%La matriz de priorización es una herramienta que mediante filas y columnas ayuda a comparar y seleccionar las prioridades entre diversas opciones o posibles soluciones, lo que hace que tomar una decisión sea mucho más sencillo. De esta forma, el diagrama de priorización permite evaluar diferentes alternativas u opciones puntuándolas respecto a criterios de interés para un problema.\citep{matrizPriorizacion}\\
Después de identificar y agrupar los problemas técnicos mediante la lluvia de ideas y el diagrama de afinidades, Ahora se debe priorizar los problemas identificados para saber que resolver primero. La matriz de priorización es utilizada para determinar qué problemas deben ser abordados primero, basándonos en su impacto y viabilidad.\\

En esta sección, se presenta la aplicación de la matriz de priorización, utilizando datos de una encuesta realizada a seis empleados directamente involucrados. Este enfoque garantiza que las decisiones de mejora se tomen de manera informada y estratégica, optimizando así el uso de los recursos disponibles.

%tabla 2 para referenciar
\begin{table}[H]
\centering
\caption[Matriz de Priorización]{Matriz de Priorización}
\begin{adjustbox}{margin=6.4cm 0cm 0cm 0cm,center} % Ajuste de márgenes: izquierda, abajo, derecha, arriba
\begin{threeparttable}
\begin{tblr}{
  row{1} = {ChetwodeBlue},
  hlines,
  vlines,
}
Ideas Base & Frec\textsuperscript{1} & Imp\textsuperscript{2} & Fact\textsuperscript{3} & Total\\
Registro y Gestión de Datos & 30 & 26 & 30 & 86\\
Gestión de Recursos Humanos & 18 & 26 & 20 & 62\\
Infraestructura y Equipamiento & 20 & 30 & 20 & 64
\end{tblr}
\begin{tablenotes}
    \vspace{0.2cm}
    \footnotesize
    \item[1]Frecuencia
    \item[2]Impacto.
    \item[3]Factibilidad.
\end{tablenotes}
\end{threeparttable}
\end{adjustbox}
\end{table}

Basándonos en la matriz de prioridades, se puede concluir que el problema principal identificado en la empresa Automotriz Pelayo S.A.C. se relaciona con el \textbf{Registro y Gestión de Datos}. Actualmente, la información de los clientes y los servicios ofrecidos no está eficientemente organizada ni centralizada, lo que dificulta su acceso y gestión por parte del personal. La ineficiencie del sistema actual también afecta a la la facturación y aumenta de los tiempos de espera para los clientes. Otras concecuencias son:
%Esté área obtuvo el puntáje más alto en los criterios de frecuencia (28), importancia(24), factibilidad, sumando un total de 82, es decir, que los problemas relacionados con la ineficiencia del sistema actual de registro y gestión de datos, es el problema más urgente a resolver.
\begin{itemize}
    \item Ineficiencia en la búsqueda de información, lo que resulta en tiempos de respuesta prolongados y retrasos en la entrega de servicios.
    \item Duplicidad de registros y posibles errores en la documentación.
    \item Dificultades para ofrecer un servicio al cliente coherente y eficiente debido a la falta de acceso rápido a la información relevante.
\end{itemize}

%Especificar el problema, 


\subsection{2.2 Objetivos del Proyecto de Mejora}
Para abordar los problemas presentes en la empresa, es necesario establecer objetivos generales y específicos que guíen las acciones del proyecto de mejora. En esta sección se definirán estos objetivos que se orientaran a resolver el problema principal.

\subsubsection*{Objetivo General:}
Optimizar el registro y gestión de datos de la empresa.
\subsubsection*{Objetivos Específicos: } 
    \begin{itemize}
        \item Optimizar el registro de datos:
        \begin{itemize}
            \item Asegurar que la información de los clientes y servicios esté correctamente registrada para facilitar su acceso y gestión.
            \item Eliminar la duplicidad de registros y mejorar la precisión de la información almacenada.
        \end{itemize}        
        
        \item Mejorar la eficiencia en la gestión de datos:
        \begin{itemize}
            \item Implementar prácticas que permitan una búsqueda y recuperación más eficiente de la información.
            \item Garantizar la coherencia y exactitud de los datos utilizados en todos los procesos operativos.
        \end{itemize}        

        \item Reducir los errores en la facturación:
        \begin{itemize}
            \item Agilizar el proceso de facturación para minimizar los retrasos en la emisión de facturas.
            \item Aumentar la precisión en la facturación para evitar errores y disputas con los clientes.
        \end{itemize}        
        
        \item Fortalecer la calidad del servicio al cliente:
        \begin{itemize}
            \item Asegurar que los datos de los clientes sean fácilmente accesibles para mejorar la respuesta a consultas y solicitudes.
            \item Mantener actualizada la información del cliente para proporcionar un servicio más personalizado y eficiente.
        \end{itemize}        
    \end{itemize}
    
\subsection{2.3 Antecedentes del Proyecto de Mejora (Investigaciones realizadas): }
Para el desarrollo efectivo de un proyecto de mejora, es esencial analizar y comprender los antecedentes que han influido en el campo de estudio. Estos antecedentes incluyen  trabajos previos, investigaciones y estudios que están directamente relacionados con los objetivos y el alcance del proyecto en curso.
%En el proceso de desarrollo de un proyecto de mejora, es crucial comprender y analizar los antecedentes que han influido en el ámbito de estudio. Estos antecedentes abarcan una amplia gama de trabajos previos, investigaciones y estudios que guardan relación con los objetivos y el alcance del proyecto actual. Constituyen una base de conocimientos y experiencias que proporcionan un contexto valioso para la formulación de estrategias y la toma de decisiones.
En esta sección, se llevará a cabo un análisis de los antecedentes relevantes, reconociendo la contribución de cada trabajo citado. 

\subsubsection*{Antecedente 1:}
%empezar con la cita.
En un estudio realizado en un taller automotriz ubicado en Lizardo García 2614, entre Colombia y Venezuela, se identificaron varios problemas relacionados con la gestión de sus procesos, los cuales se manejaban utilizando hojas de Excel . Este enfoque causaba una serie de conflictos, tales como pérdidas económicas debido a la falta de un registro automatizado de los ingresos y salidas de materiales, y pérdida de tiempo para los clientes que solicitaban informes de sus últimas visitas . Además, en caso de presentarse algún inconveniente respecto al servicio prestado, no era posible identificar de inmediato a los responsables, lo que generaba descontento tanto en los clientes como en el propietario del negocio con respecto a sus empleados.\\
Para abordar estos problemas, se propuso la tesis titulada “Análisis y Desarrollo de un Sistema de Control y Ficha Técnica de Taller Automotriz”, que presentó una solución basada en una aplicación web. Esta aplicación se diseñó para mejorar la metodología de control y seguimiento de los procesos del taller, apoyando aspectos clave como la facturación, la petición, la venta y la adquisición de productos necesarios para brindar un servicio eficiente a los clientes . La investigación se fundamentó en una metodología descriptiva, con técnicas de campo aplicadas a una población de 57 empleados del taller, de los cuales se tomó una muestra de 50 . La implementación del sistema se llevó a cabo utilizando Visual Studio 2010, el lenguaje de programación C\#, y la base de datos SQL Server 2008 R2, complementados con CSS y HTML para mejorar la visualización .
La mejora en la calidad del servicio derivada de este sistema benefició principalmente a los clientes y, por ende, a la empresa en su conjunto .

\begin{flushright}
{Fuente: \citep{antecedente1Tesis}}
\end{flushright}
%Para el estudio realizado se trabaja con un taller automotriz ubicada en Lizardo García 2614 entre Colombia y Venezuela se encargan de brindar servicios automotriz a la ciudadanía en general estos manejan sus procesos empleando hojas de Excel que les ocasiona ciertos conflictos como:
%Pérdida económicas puesto que no se posee un registro automatizado de los ingresos y salidas de materiales. Pérdida de tiempo por parte del cliente, en caso de que este solicite un informe de sus últimas visitas. En caso de presentarse algún inconveniente con respecto al servicio prestado, no se podrá conocer de forma inmediata los responsables generando con esto un descontento por parte del cliente y del dueño del negocio para con sus empleados. Observando cada uno de estos puntos y luego de un análisis e interpretación de los datos de investigación se presenta la propuesta de tesis Análisis y Desarrollo de un Sistema de Control y Ficha Técnica de Taller Automotriz, la aplicación web establece una metodología para el control y seguimiento de los procesos que se llevan a cabo dentro de un local de servicios para apoyar el tema de facturación, petición, venta y adquisición de productos involucrados en brindar atención de manera eficiente a los clientes. El estudio se basa en una metodología de investigación descriptiva, con técnicas de campo, trabajando sobre la población de los empleados del taller automotriz que son alrededor de 57 empleados, y como muestra 50 de estos. Al mejorar la calidad de servicio que brinda el establecimiento los principales beneficiarios serán los clientes y con esto la empresa en general.
%Para la realización del sistema se emplea Visual Studio 2010, lenguaje de programación C\#, y como base de datos SQL Server 2008 R2, y para mejorar la visualización del mismo se trabaja con el apoyo de CSS y HTML.\noindent \citep{antecedente1}

\subsubsection*{Antecedente 2:}
%incluir la cita antes.
%El proyecto de fin de carrera "Gestión web de un taller mecánico" se ha desarrollado para la empresa Talleres Tauro y Richard, con sede en Galdakao. El objetivo del proyecto es implantar una aplicación web, con un diseño elaborado, para que los clientes puedan acceder a toda la información referente a esta empresa. Esta aplicación cumplirá totalmente con la legislación española, puesto que posee aviso legal, política de rivacidad y aviso de cookies, con su correspondiente política de cookies.
%Para la realización de una web de tal magnitud, es necesario el uso de gestores de ontenidos y se ha optado por la utilización de WordPress, cuyo número de usuarios que lo utiliza aumenta exponencialmente, debido a las numerosas prestaciones que tiene y a lo sencillo que es acceder a su código. 
%Una de las peculiaridades de esta aplicación web es que, además de mostrar toda la información referente a la empresa, posee un módulo, que se diseñará y desarrollará en este trabajo fin de grado, que permitirá a todos los clientes de la empresa poder saber, en cada instante, el estado de su automóvil y pagar la factura de la reparación desde casa, con total confianza y comodidad, y todo ello mediante PayPal o por transferencia bancaria.
%Otro de los módulos que se va a desarrollar va a ser el panel de administración, a través del cual los administradores podrán realizar numerosas gestiones, así como, introducir un nuevo vehículo, modificar datos de vehículos o reparaciones existentes, ver el historial de todas las reparaciones que se han realizado en dicho taller, pudiendo ordenarlas por fecha o nombre y gestionar los servicios de mano de obra que irán incluidos en las facturas, pudiendo insertar, modificar o eliminar los que se desee. Además el administrador también podrá visualizar las cámaras web del taller desde esta aplicación y podrá insertar ofertas que aparecerán visibles en la web para todos los usuarios, las cuales se publicarán, a su vez, en el muro del Facebook del taller.
%Por último, el administrador podrá añadir o eliminar avisos, que servirán para estar conectados con la App móvil para Android, que también se va a desarrollar a lo largo de este trabajo fin de grado. Esta App permitirá a los clientes, que dispongan de ella, saber si la reparación de su vehículo ha finalizado, pedir cita a través de la misma e, incluso, saber con cuantos kilómetros debe cambiar el aceite o las pastillas de frenos o cual es la fecha de la próxima ITV, entre otras funcionalidades\citep{antecedente2}.
En el proyecto de fin de carrera titulado "Gestión web de un taller mecánico", desarrollado para la empresa Talleres Tauro y Richard en Galdakao, se propuso la creación de una aplicación web que permitiría a los clientes acceder a toda la información relevante sobre la empresa y sus servicios . Esta aplicación, diseñada para cumplir con la legislación española en términos de aviso legal, política de privacidad y cookies, fue implementada utilizando WordPress debido a su facilidad de uso y la creciente comunidad de usuarios que lo respaldan . El proyecto, descrito en \citep{antecedente2Tesis}, destaca por su enfoque en mejorar la interacción con los clientes y optimizar la gestión interna del taller.

Una característica distintiva de esta aplicación web es un módulo que permite a los clientes verificar el estado de sus vehículos y pagar las facturas de reparación en línea mediante PayPal o transferencia bancaria, lo que proporciona una mayor comodidad y confianza . Además, se desarrolló un panel de administración que permite a los administradores realizar diversas gestiones, como la inserción y modificación de datos de vehículos, la visualización del historial de reparaciones y la gestión de los servicios de mano de obra incluidos en las facturas . La aplicación también ofrece la capacidad de visualizar las cámaras web del taller, publicar ofertas visibles para todos los usuarios y conectarse con una aplicación móvil para Android, facilitando la comunicación y gestión de citas .

Este enfoque integral, detallado en el proyecto \citep{antecedente2Tesis}, no solo mejora la experiencia del cliente al proporcionar acceso instantáneo a la información y servicios del taller, sino que también optimiza la administración interna, permitiendo una gestión más eficiente de los recursos y procesos .

\begin{flushright}
{Fuente: \citep{antecedente2Tesis}}
\end{flushright}



\subsection{2.4 Justificación del Proyecto de Mejora: }
%de acuerdo a los antecedentes - que falta el texto de justificación
%La empresa \textbf{AUTOMOTRIZ PELAYO S.A.C.} ha entrado en el mercado local  en servicios de mantenimiento mecánico y reparación de vehículos y maquinaria. Este proyecto de mejora se centra en la resolución de problemas críticos relacionados con el registro y gestión de datos, lo que es fundamental para la calidad del servicio y la operatividad de la empresa.\\
%Es necesario efectuar este proyecto debido a que los problemas actuales en la gestión de datos impactan directamente en la precisión y rapidez del servicio al cliente. En un mercado cada vez más competitivo, la capacidad de manejar la información de manera eficiente y sin errores es importante para la toma de decisiones tanto económicas como operativas. Mejorar estos procesos permitirá reducir errores, agilizar la facturación, y garantizar la disponibilidad precisa y oportuna de la información, mejorando así la experiencia del cliente y la productividad del personal.
La empresa \textbf{AUTOMOTRIZ PELAYO S.A.C.} se ha establecido en el mercado local como un proveedor clave de servicios de mantenimiento mecánico y reparación de vehículos y maquinaria. En un contexto de alta competencia y exigencia por parte de los clientes, la precisión y la eficiencia en la gestión de datos son cruciales para asegurar un servicio de calidad y mantener la operatividad de la empresa. Este proyecto de mejora se enfoca en abordar los problemas críticos relacionados con el registro y gestión de datos, que actualmente afectan de manera significativa la eficiencia operativa y la satisfacción del cliente.
Los problemas actuales de \textbf{AUTOMOTRIZ PELAYO S.A.C.} en la gestión de datos, como la ineficiencia en la búsqueda de información, la duplicidad de registros y la dificultad en el acceso a datos precisos y actualizados, están causando varias repercusiones negativas. Estas incluyen la pérdida de tiempo, la frustración de los clientes debido a retrasos en la obtención de información sobre servicios pasados y la incapacidad de rastrear rápidamente las responsabilidades en caso de disputas sobre servicios prestados. Estos problemas no solo afectan la operatividad diaria, sino que también pueden dañar la reputación de la empresa y su posición competitiva en el mercado.
El proyecto de mejora es crucial para \textbf{AUTOMOTRIZ PELAYO S.A.C.} por las siguientes razones:
\begin{enumerate}
    \item Mejora de la Precisión y Eficiencia Operativa: La implementación de un sistema de gestión de datos más eficiente permitirá reducir significativamente los errores en los registros y la duplicación de información. Esto agilizará los procesos internos y mejorará la precisión en la facturación lo que es vital para mantener una operatividad fluida y eficiente.
    \item Aumento de la Satisfacción del Cliente: Al garantizar una disponibilidad precisa y oportuna de la información, la empresa podrá ofrecer respuestas rápidas y exactas a las consultas de los clientes sobre sus vehículos y servicios anteriores. Esto no solo reducirá el tiempo de espera , sino que también aumentará la confianza y satisfacción de los clientes, fomentando la fidelidad y la repetición del negocio.
    \item Fortalecimiento de la Competitividad: En un mercado competitivo, la capacidad de manejar información de manera eficiente y sin errores es una ventaja competitiva crucial. Al mejorar los procesos de gestión de datos, \textbf{AUTOMOTRIZ PELAYO S.A.C.} estará mejor posicionada para tomar decisiones estratégicas basadas en datos precisos, responder de manera más eficaz a las demandas del mercado y ofrecer un servicio de mayor calidad que sus competidores.
    \item Optimización de Recursos: Un sistema de gestión de datos más eficaz permitirá una mejor utilización de los recursos humanos y materiales de la empresa. Al reducir el tiempo y el esfuerzo dedicados a la gestión de datos ineficaz, el personal podrá enfocarse en tareas de mayor valor añadido, como la mejora continua de los servicios y la atención al cliente.
    \item Mejora de la Toma de Decisiones: Disponer de datos precisos y accesibles es fundamental para la toma de decisiones informadas en cualquier negocio. Al resolver los problemas actuales de gestión de datos, la empresa podrá tomar decisiones más acertadas y oportunas en cuanto a la adquisición de materiales, la planificación de servicios y la gestión de recursos, lo que contribuirá a una mejor gestión y crecimiento del negocio.
\end{enumerate}


\subsection{2.5 Marco Teórico y Conceptual}
En esta sección, se establece el marco teórico y conceptual necesario para comprender los fundamentos esenciales que guiarán el desarrollo y la implementación del proyecto de mejora. Se exploran temas clave que van desde la gestión eficiente de bases de datos con SQL hasta el diseño de interfaces dinámicas con Angular. Además, se presentan conceptos fundamentales como el desarrollo en el lado del servidor (Backend), la interacción del usuario con la aplicación (Frontend), y herramientas modernas como Spring Boot y Angular que facilitan la creación de aplicaciones robustas y escalables.

\subsubsection{2.5.1 Fundamento teórico del Proyecto de Mejora}
En esta subsección se discuten los fundamentos teóricos que sustentan el proyecto de mejora, abordando aspectos esenciales como SQL y bases de datos, el rol crucial del Backend y Frontend en el desarrollo web, y la eficiencia operativa que se busca optimizar a través de tecnologías como Spring Boot y Java.

\subsubsection{SQL y Bases de Datos:}
\paragraph{SQL (Structured Query Language). }
SQL es un lenguaje de programación estándar utilizado para gestionar y manipular bases de datos relacionales. Es fundamental en la administración de datos porque permite realizar diversas operaciones como la inserción, actualización, eliminación y consulta de datos en una base de datos. Las bases de datos relacionales, como MySQL o PostgreSQL, organizan los datos en tablas interrelacionadas, lo que facilita la gestión eficiente y la integridad de los datos.

\paragraph{Bases de Datos Relacionales.}
Una base de datos relacional utiliza un esquema tabular para definir datos y sus relaciones. Cada tabla contiene filas (registros) y columnas (atributos), y las relaciones entre tablas se establecen mediante claves primarias y foráneas. Esto permite consultas complejas y la manipulación de datos de manera estructurada y lógica, asegurando la integridad y coherencia de la información.

\paragraph{Bases de Datos No Relacionales.}
A diferencia de las bases de datos relacionales, las bases de datos no relacionales (o NoSQL) no utilizan un esquema tabular predefinido. En su lugar, están diseñadas para almacenar y recuperar datos de manera flexible y escalable, adaptándose mejor a las necesidades de grandes volúmenes de datos no estructurados o semi-estructurados, como los datos generados por aplicaciones web, redes sociales y dispositivos IoT.

\subsubsection{Backend y Frontend}
\paragraph{Backend.}
Es la parte del desarrollo web que gestiona la lógica del servidor, las bases de datos y la aplicación en su conjunto. Incluye todas las operaciones que ocurren en el "detrás de escena" y que son esenciales para el funcionamiento de la aplicación. Los desarrolladores de backend se enfocan en la creación de sistemas robustos y escalables, utilizando lenguajes de programación como Java, Python o Node.js, y gestionando bases de datos, servidores y APIs.

\paragraph{Frontend.}
Es la interfaz de usuario y todo lo que los usuarios ven e interactúan en una aplicación web. Los desarrolladores de frontend utilizan tecnologías como HTML, CSS y JavaScript para crear interfaces atractivas y funcionales que mejoren la experiencia del usuario. Frameworks como Angular, React y Vue.js son comunes en el desarrollo de frontend, facilitando la creación de aplicaciones web dinámicas y responsivas.

\paragraph{Diferencias entre Backend y Frontend.}
La principal diferencia entre el backend y el frontend radica en su enfoque y función dentro de una aplicación. El backend se ocupa de la lógica empresarial, la gestión de datos y la seguridad, proporcionando la infraestructura necesaria para que la aplicación funcione correctamente. Por otro lado, el frontend se encarga de la experiencia del usuario, diseñando y desarrollando la interfaz visual y la interacción del usuario con la aplicación. Mientras el backend se centra en el procesamiento y almacenamiento de datos, el frontend se dedica a la presentación y la experiencia del usuario final.

\subsubsection{Spring Boot y Java}
\paragraph{Java}
Es un lenguaje de programación robusto y ampliamente utilizado para el desarrollo de aplicaciones empresariales, móviles y web. Su portabilidad, seguridad y rendimiento lo hacen ideal para construir sistemas grandes y complejos. Java es el lenguaje subyacente para muchos frameworks y plataformas de desarrollo, incluyendo Spring .

\paragraph{Spring Boot.}
Es una extensión del framework Spring que facilita la creación de aplicaciones Java autónomas y listas para producción con configuración mínima. Simplifica la configuración y despliegue de aplicaciones al proporcionar un conjunto de herramientas que automatizan la mayoría de las tareas necesarias, como la configuración del servidor y la gestión de dependencias. Es especialmente útil para desarrollar microservicios y aplicaciones web modernas.

\subsubsection{ORM (Object-Relational Mapping)}
El ORM es una técnica de programación que permite convertir datos entre sistemas incompatibles en una base de datos relacional y objetos de programación orientados a objetos. Esta técnica simplifica y agiliza el desarrollo al abstraer el acceso a la base de datos, permitiendo a los desarrolladores interactuar con la base de datos a través de objetos en lugar de SQL. Los ORM como Hibernate en Java y Entity Framework en .NET son herramientas comunes que facilitan esta tarea, mejorando la productividad y reduciendo el riesgo de errores.

\subsubsection{API (Application Programming Interface) y REST (Representational State Transfer)}
\paragraph{API.}
Es un conjunto de reglas y protocolos que permiten a diferentes software comunicarse entre sí. Facilita la integración de diferentes sistemas y servicios, permitiendo el acceso y uso de funcionalidades y datos de una aplicación desde otra. Las APIs son esenciales para la interoperabilidad y la extensión de las aplicaciones, permitiendo que terceros desarrolladores creen aplicaciones que se integren con la plataforma.

\paragraph{REST.}
Es un estilo de arquitectura para diseñar redes de aplicaciones, que utiliza el protocolo HTTP para la comunicación entre sistemas. Las APIs RESTful son fáciles de integrar y escalar, y son ampliamente utilizadas en el desarrollo de aplicaciones web y servicios micro. Estas APIs permiten realizar operaciones como obtener, crear, actualizar y eliminar datos a través de simples solicitudes HTTP, facilitando la interoperabilidad entre sistemas.

\paragraph{Arquitectura de REST API.}
La arquitectura REST API se basa en la idea de recursos, donde cada recurso es identificado por una URL única. Los recursos pueden ser manipulados utilizando los métodos estándar de HTTP como GET, POST, PUT y DELETE. Estas operaciones permiten la creación, lectura, actualización y eliminación de recursos, respetando los principios de la transferencia de estado representacional (REST). Además, REST API promueve la escalabilidad y la interoperabilidad mediante la separación clara entre el cliente y el servidor, permitiendo que diferentes sistemas y aplicaciones interactúen de manera flexible y eficiente.

\subsubsection{Angular}
Angular es una plataforma y un framework para crear aplicaciones de una sola página en el lado del cliente usando HTML y TypeScript. Angular está escrito en TypeScript e implementa la funcionalidad básica y opcional como un conjunto de bibliotecas TypeScript que importas en tus aplicaciones. Angular facilita el desarrollo de aplicaciones web dinámicas y responsivas, ofreciendo una arquitectura estructurada que permite el desarrollo modular y reutilizable.

\paragraph{Componentes en Angular.}
Los componentes son la base fundamental de las aplicaciones en Angular. Son bloques de construcción reutilizables y autónomos que encapsulan la lógica, la plantilla y los estilos de una parte específica de la interfaz de usuario. Cada componente se compone de una clase que gestiona la lógica de la aplicación y una plantilla que define la vista, permitiendo la creación de interfaces de usuario complejas a partir de piezas más pequeñas y manejables.

\paragraph{Directivas en Angular.}
Las directivas son funciones que se ejecutan cada vez que el compilador Angular las encuentra. Las directivas angulares mejoran la capacidad de los elementos HTML al adjuntar comportamientos personalizados al DOM. Existen tres tipos principales de directivas: directivas estructurales, directivas de atributo y directivas de clase. Angular proporciona directivas integradas como *ngIf y *ngFor, que son ampliamente utilizadas para la lógica condicional y la iteración en las plantillas.



%explicar cosas sql, base de datos, tipos, backend, front end, diferencias



\subsubsection{2.5.2 Conceptos y términos utilizados}
Esta parte del marco teórico y conceptual proporciona una lista detallada de conceptos clave y términos técnicos que serán utilizados a lo largo del proyecto. Desde la inyección de dependencias y el patrón MVC hasta herramientas como Git y GitHub, cada término se define de manera sencilla para garantizar una comprensión clara y precisa.

\begin{enumerate}
    \item Eficiencia Operativa: La eficiencia operativa se refiere a la capacidad de un equipo para entregar un producto de calidad, con la menor cantidad de recursos posible. Se mide calculando la relación entre lo que inviertes en el proyecto, a menudo denominado recursos, y los resultados obtenidos, denominados productos.\citep{asanaWantMore}
    \item Satisfacción del Cliente: En el mundo altamente competitivo en el que vivimos, la satisfacción del cliente es esencial para las empresas. No importa el rubro al que pertenezcas, ya no basta con llegar primero al mercado o con contratar al publicista de moda. Los tiempos han cambiado y con ello la forma en la que los consumidores piensan, y esto nos lleva a que hemos modificado los hábitos de compra. El consumidor actualmente tiene una elección difícil a la hora de adquirir un producto o servicio, ya que delante de él se encuentran 50 marcas del mismo tipo que buscan su preferencia, pero, ¿cómo lograr que consuman la tuya? la respuesta es sencilla: Lograr la satisfacción del cliente.\citep{questionproSatisfaccinCliente}
    \item Backend: El backend es un término que utilizamos para referirnos a la arquitectura interna de un sitio web. Esta área lógica, que no es visible a los ojos del usuario y no incluye elementos de tipo gráfico, permite que todos los elementos de una web desarrollen la función correcta.
    La palabra, en inglés, se refiere a ‘lo que está detrás’, ‘lo que no se ve’. De ahí su término ‘back’. Cuando hablamos de backend, hablamos únicamente del código interno de la página. Quien lo desarrolla, desarrolla la funcionalidad del sitio y la seguridad y la optimización de los recursos.\citep{devcampQuBackend}
    \item Spring Boot: Java Spring Framework (Spring Framework) es una popular estructura empresarial de código abierto para crear aplicaciones independientes de nivel de producción que se ejecutan en la máquina virtual Java (JVM).
    Java Spring Boot (Spring Boot) es una herramienta que hace que el desarrollo de aplicaciones web y microservicios con Spring Framework sea más rápido y fácil.\citep{ibmQuJava}
    \item SQL (Structured Query Language): El lenguaje de consultas estructuradas o SQL (Structured Query Language) es un lenguaje de programación estandarizado que se utiliza para administrar bases de datos relacionales y realizar diversas operaciones con los datos que contienen. Creado inicialmente en la década de 1970, SQL es utilizado habitualmente no solo por los administradores de bases de datos, sino también por los desarrolladores que escriben scripts de integración de datos y por los analistas de datos que desean configurar y ejecutar consultas analíticas.\citep{computerweeklyQuStructured}
    \item Base de datos relacional: Una base de datos relacional es una colección de información que organiza datos en relaciones predefinidas, en la que los datos se almacenan en una o más tablas (o "relaciones") de columnas y filas, lo que facilita su visualización y la comprensión de cómo se relacionan las diferentes estructuras de datos entre sí. Las relaciones son conexiones lógicas entre las diferentes tablas y se establecen a partir de la interacción entre ellas.\citep{googleQuBase}
    \item MySQL: MySQL es un sistema de administración de bases de datos relacionales. Es un software de código abierto desarrollado por Oracle. Se considera como la base de datos de código abierto más utilizada en el mundo.\citep{hubspotMySQLPara}
    \item ORM (Object-Relational Mapping): Un ORM es un modelo de programación que permite mapear las estructuras de una base de datos relacional (SQL Server, Oracle, MySQL, etc.), en adelante RDBMS (Relational Database Management System), sobre una estructura lógica de entidades con el objeto de simplificar y acelerar el desarrollo de nuestras aplicaciones.\citep{deloitteQuORM}
    \item API (Application Programming Interface): La Application Programming Interface (API) es un conjunto de patrones que forman parte de una interfaz que permite la creación de plataformas de una forma más sencilla y práctica para desarrolladores. Aprende más a continuación.\citep{deloitteQuORM}
    \item REST (Representational State Transfer): REST es una interfaz para conectar varios sistemas basados en el protocolo HTTP (uno de los protocolos más antiguos) y nos sirve para obtener y generar datos y operaciones, devolviendo esos datos en formatos muy específicos, como XML y JSON.\citep{openwebinarsRESTConoce}
    \item MVC (Model-View-Controller): El MVC o Modelo-Vista-Controlador es un patrón de arquitectura de software que, utilizando 3 componentes (Vistas, Models y Controladores) separa la lógica de la aplicación de la lógica de la vista en una aplicación. Es una arquitectura importante puesto que se utiliza tanto en componentes gráficos básicos hasta sistemas empresariales; la mayoría de los frameworks modernos utilizan MVC (o alguna adaptación del MVC) para la arquitectura, entre ellos podemos mencionar a Ruby on Rails, Django, AngularJS y muchos otros más. En este pequeño artículo intentamos introducirte a los conceptos del MVC.\citep{codigofacilitoModelView}
    \item DTO (Data Transfer Object): Una de las problemáticas más comunes cuando desarrollamos aplicaciones, es diseñar la forma en que la información debe viajar desde la capa de servicios a las aplicaciones o capa de presentación, ya que muchas veces por desconocimiento o pereza, utilizamos las clases de entidades para retornar los datos, lo que ocasiona que retornemos más datos de los necesarios o incluso, tengamos que ir en más de una ocasión a la capa de servicios para recuperar los datos requeridos.
    El patrón DTO tiene como finalidad de crear un objeto plano (POJO) con una serie de atributos que puedan ser enviados o recuperados del servidor en una sola invocación, de tal forma que un DTO puede contener información de múltiples fuentes o tablas y concentrarlas en una única clase simple.\citep{oscarblancarteblogDataTransfer}
    \item Repositorio de código: Un repositorio, o repo, es un tipo de almacenamiento digital centralizado que los desarrolladores utilizan para realizar y administrar cambios en el código fuente de una aplicación. Los desarrolladores tienen que almacenar y compartir carpetas, archivos de texto y otros tipos de documentos al desarrollar software. Un repositorio cuenta con características que permiten a los desarrolladores rastrear con facilidad cambios en el código, editar archivos de manera simultánea y colaborar de forma eficiente en el mismo proyecto desde cualquier ubicación.\citep{amazonQuCLI}
    \item Git: Git es un sistema de control de versiones distribuido, lo que significa que un clon local del proyecto es un repositorio de control de versiones completo. Estos repositorios locales plenamente funcionales permiten trabajar sin conexión o de forma remota con facilidad. Los desarrolladores confirman su trabajo localmente y, a continuación, sincronizan la copia del repositorio con la del servidor. Este paradigma es distinto del control de versiones centralizado, donde los clientes deben sincronizar el código con un servidor antes de crear nuevas versiones.\citep{microsoftQuGit}
    \item GitHub: GitHub es una herramienta esencial para los ingenieros de software, y su popularidad es inigualable. Actualmente cuenta con más de 25 millones de usuarios. Se trata de un número considerable de profesionales que recurren a GitHub para mejorar el flujo de trabajo y la colaboración.\citep{hostingerGitHubCmo}
    \item Servidor de aplicaciones: En sistemas que son cada vez más grandes, necesitas soluciones que puedan asumir el volumen de datos, manteniendo la velocidad que deseas y al mismo tiempo prestando servicio al volumen de acceso. En una red cliente-servidor, un servidor de aplicaciones puede ser una buena opción. Un servidor de aplicaciones suele alojar distintos programas de aplicación y los pone a disposición de los clientes. Para ello, utiliza la lógica empresarial del lado del servidor para generar contenido dinámico y mostrarlo al cliente. Algunos ejemplos típicos de software que se encuentran en un servidor de aplicaciones son los programas ofimáticos, la gestión de direcciones, los calendarios corporativos y el acceso a bases de datos. Los procesos de carácter confidencial, como las transacciones o las autenticaciones, también pueden realizarse a través de un servidor de aplicaciones.\citep{ionosQuServidores}
    \item Framework: El framework es un término que proviene del inglés y significa «marco de trabajo» o «estructura». En el ámbito de la programación, un framework es un conjunto de herramientas y librerías que se utilizan para desarrollar aplicaciones más fácilmente y de manera más eficiente. 
    Un framework es un conjunto de reglas y convenciones que se usan para desarrollar software de manera más eficiente y rápida. Estos marcos de trabajo se emplean para ahorrar tiempo y esfuerzo en el desarrollo de aplicaciones, ya que proporcionan una estructura básica que se puede utilizar como punto de partida. Además, los frameworks también ofrecen soluciones a problemas comunes en el desarrollo de software, lo que significa que los desarrolladores pueden centrarse en las funcionalidades específicas de su aplicación en lugar de perder tiempo resolviendo problemas técnicos.\citep{cesumaQuFramework}
    \item Inyección de dependencias: La inversión de dependencias es un principio que describe un conjunto de técnicas destinadas a disminuir el acoplamiento entre los componentes de una aplicación. Es uno de los principios SOLID más populares y utilizados en la creación de aplicaciones, frameworks y componentes por las ventajas que aporta a las mismas.\citep{campusmvpInyeccinDependencias}
    \item Frontend: El Frontend, es la fachada de toda experiencia en línea. Es la "magia" detrás de las interfaces visuales que navegamos a diario: desde los colores y las imágenes hasta los botones y los menús. En esencia, es el artista que da vida a la tecnología, permitiendo que los usuarios interactúen con sitios web y aplicaciones de manera intuitiva y atractiva, ocupando un rol muy importante en la experiencia del usuario.\citep{linkedinFrontendCaractersticas}
    \item Angular: Angular es una plataforma y un framework para crear aplicaciones de una sola página en el lado del cliente usando HTML y TypeScript. Angular está escrito en TypeScript. Implementa la funcionalidad básica y opcional como un conjunto de bibliotecas TypeScript que importas en tus aplicaciones.\citep{angularAngular}
    \item Componentes (en el contexto de Angular): Los componentes son la base fundamental de las aplicaciones en Angular. Son bloques de construcción reutilizables y autónomos que encapsulan la lógica, la plantilla y los estilos de una parte específica de la interfaz de usuario. En este artículo, exploraremos qué son los componentes en Angular y cómo crearlos.\citep{QuComponentes}
    \item Directivas (en el contexto de Angular): Las directivas son las funciones que se ejecutarán cada vez que el compilador Angular las encuentre. Las directivas angulares mejoran la capacidad de los elementos HTML al adjuntar comportamientos personalizados al DOM.\citep{freecodecampCmoUsar}
    \item Servicios (en el contexto de Angular): Un servicio en Angular es una clase con una funcionalidad específica que se utiliza para compartir datos, lógica de negocio y funcionalidades entre diferentes componentes de una aplicación. El propósito principal de un servicio es proporcionar una capa de abstracción y modularidad, permitiendo la reutilización del código y facilitando la separación de preocupaciones.
    Los servicios en Angular se utilizan comúnmente para realizar llamadas a API, gestionar el estado de la aplicación, compartir datos entre componentes, manejar la autenticación y autorización, y mucho más. Al encapsular la lógica en servicios, podemos mantener nuestros componentes más livianos y centrados en la presentación, lo que mejora la legibilidad, mantenibilidad y escalabilidad del código.\citep{imaginaformacionCmoIntegrar}
    \item SPA (Single Page Application): Una Single-Page Application (SPA) es un tipo de aplicación web que ejecuta todo su contenido en una sola página. 
    Funciona cargando el contenido HTML, CSS y JavaScript por completo al abrir la web. Al ir pasando de una sección a otra, solo necesita cargar el contenido nuevo de forma dinámica si este lo requiere, pero no hace falta cargar la página por completo. Esto mejora los tiempos de respuesta y agiliza mucho la navegación, favoreciendo así a la experiencia de usuario.\citep{digital55QuSinglePage}
    \item CLI (Command Line Interface): Una interfaz de la línea de comandos (CLI) es un mecanismo de software que se utiliza para interactuar con el sistema operativo mediante el teclado. Otro mecanismo disponible es la interfaz de usuario gráfica (GUI), la cual se utiliza mucho en la actualidad en todas las aplicaciones y los sistemas de software. Puede usar una GUI para navegar visualmente y hacer clic en íconos e imágenes a fin de poner actividades en funcionamiento. Sin embargo, las GUI resultan ineficientes para las tareas de administración del sistema, en especial cuando se trata de entornos virtuales o remotos. Con una interfaz de línea de comandos, puede escribir comandos de texto para configurar, explorar o ejecutar programas en cualquier servidor o sistema informático. Todos los sistemas operativos, incluidos Linux, macOS y Windows, proporcionan una CLI para agilizar la interacción con el sistema.\citep{amazonQuCLI}
    \item Routing (en el contexto de Angular): Enrutamiento o rutas en Angular es la manera en la que navegamos entre las vistas de nuestra aplicación, en una web normal nosotros navegamos entre paginas HTML, pero en Angular navegamos entre vistas que hemos generado a base de módulos y componentes.\citep{mediumEnrutamientoAngular}
\end{enumerate}






%Capitulo 2
%
%Revisar el manual para el desarrollo del índice de Identificación del problema
%
%Falta desarrollar mas, indicar las consultas o entervistas o cuestionarios desarrollados para obtener la información.
%
%Falta desarrollar un cuadro de priorizacion segun los conceptos o caracteristicas que consideren relevantes para la identificacion del problema a resolver
%
%El objetivo principal esta correcto
%
%Los objetivos secundarios NO (un objetivo nunca debe contener la solucion) Revisar la indicacion de objetivos anteriores
%
%Los antecedentes corresponden a proyectos desarrollados en el pasado de cualquier nivel considerando 2 aspectos (tecnologia o problematica) y no olvidar utilizar citas y referencias bibliograficas para cada antecedente
%
%ATENDER A LAS CLASES PUES MUCHOS DE LOS ERRORES FUERON DESARROLLADOS EN LA MISMA