\section*{RESUMEN EJECUTIVO DEL PROYECTO DE MEJORA}

El presente proyecto se centra en la optimización y automatización de la gestión de datos e información para talleres mecánicos, desarrollado en el marco del Servicio Nacional de Adiestramiento en Trabajo Industrial (SENATI) bajo la dirección del autor Eliot Roy Arias Flores y la asesoría de César Rosas Aragón. Este proyecto se realizó con el objetivo de mejorar la eficiencia operativa de la empresa Automotriz Pelayo S.A.C. mediante la implementación de un sistema de gestión de datos más eficiente.

En el primer capítulo, se identificaron los problemas técnicos que afectan el desempeño y la eficiencia operativa de la empresa. Utilizando herramientas como la lluvia de ideas, el diagrama de afinidades y la matriz de priorización, se evaluaron los problemas en función de su frecuencia, importancia y factibilidad. Los problemas identificados incluyeron la falta de disponibilidad de piezas, un deficiente sistema centralizado de registro, falta de personal capacitado, sobrecarga de tareas y deficiencias en la infraestructura y equipamiento del taller.

El segundo capítulo establece los objetivos del proyecto, tanto generales como específicos, con el objetivo principal de optimizar el registro y gestión de datos de la empresa. Los objetivos específicos incluyen asegurar la correcta registración de la información, eliminar duplicidades, mejorar la eficiencia en la gestión de datos, reducir errores en la facturación y fortalecer la calidad del servicio al cliente.

En el tercer capítulo se revisan los antecedentes del proyecto, analizando estudios previos y proyectos similares realizados en talleres mecánicos. Estos antecedentes proporcionan un contexto valioso y lecciones aprendidas que influyen en la formulación de estrategias y la toma de decisiones para el proyecto actual.

El cuarto capítulo justifica la necesidad del proyecto de mejora, destacando los problemas críticos actuales de la gestión de datos en Automotriz Pelayo S.A.C. La implementación de este proyecto es crucial para mejorar la precisión y eficiencia operativa, aumentar la satisfacción del cliente, fortalecer la competitividad de la empresa y optimizar el uso de recursos.

El quinto capítulo presenta el marco teórico y conceptual del proyecto, abordando temas clave como SQL y bases de datos, el desarrollo backend y frontend, y herramientas como Spring Boot y Angular. También se explican conceptos fundamentales relacionados con la eficiencia operativa y la satisfacción del cliente, que son esenciales para la implementación exitosa del proyecto.

El sexto capítulo detalla el plan de acción de la mejora propuesta, incluyendo las tareas, recursos necesarios, cronograma de ejecución y posibles limitaciones que podrían surgir. La planificación detallada y la asignación de responsabilidades claras son esenciales para asegurar una implementación eficiente y efectiva del proyecto.

Finalmente, el séptimo capítulo evalúa los costos de implementación, incluyendo materiales, recursos humanos, equipos y otros gastos relevantes. Este análisis proporciona una visión completa de los recursos financieros necesarios para llevar a cabo la mejora con éxito, asegurando la viabilidad y sostenibilidad del proyecto.

En conclusión, el proyecto de mejora ha logrado una optimización efectiva en el registro y gestión de datos en Automotriz Pelayo S.A.C., eliminando duplicidades, mejorando la precisión y eficiencia en la recuperación de información, y fortaleciendo la calidad del servicio al cliente.


%\subsection{Introducción}
%
%El presente proyecto se centra en la optimización y automatización de la gestión de datos e información para talleres mecánicos. Este trabajo se ha realizado en el marco del Servicio Nacional de Adiestramiento en Trabajo Industrial (SENATI), dentro de la Escuela de Tecnologías de la Información, y ha sido desarrollado por Eliot Roy Arias Flores bajo la asesoría de César Rosas Aragón.
%
%
%\subsection*{Objetivos del Proyecto}
%El objetivo general del proyecto es optimizar el registro y gestión de datos de la empresa para mejorar la eficiencia operativa. Los objetivos específicos incluyen:
%
%\begin{enumerate}
%    \item Asegurar la correcta registración de la información de clientes y servicios.
%    \item Eliminar la duplicidad de registros y mejorar la precisión de la información.
%    \item Implementar prácticas para una búsqueda y recuperación más eficiente de la información.
%    \item Agilizar y aumentar la precisión en el proceso de facturación.
%    \item Fortalecer la calidad del servicio al cliente, asegurando que los datos sean accesibles y actualizados.
%\end{enumerate}
%
%\subsection*{Metodología}
%La metodología del proyecto se basa en un análisis detallado de los problemas técnicos actuales, seguido de la identificación de objetivos de mejora, la implementación de soluciones tecnológicas, y la evaluación de los resultados obtenidos. Se han utilizado herramientas como bases de datos SQL, frameworks de desarrollo backend y frontend (Spring Boot y Angular), y técnicas de mapeo objeto-relacional y APIs REST.
%
%
%\subsection*{Resultados Esperados}
%Se espera que la implementación del proyecto resulte en un ahorro significativo de tiempo y recursos, mejorando la eficiencia en la gestión de datos y la calidad del servicio al cliente. Los beneficios económicos se derivarán principalmente del ahorro en mano de obra y la reducción de errores en la facturación.
%
%
%\subsection*{Conclusiones}
%El proyecto ha concluido con éxito, cumpliendo con los objetivos propuestos. Se ha logrado una optimización efectiva en el registro y gestión de datos, eliminando duplicidades y mejorando la precisión y eficiencia en la recuperación de información. Además, se ha fortalecido la calidad del servicio al cliente, proporcionando un acceso más rápido y preciso a la información relevante.