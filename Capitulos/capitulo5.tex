\newpage
\section{CAPITULO V}
\section*{COSTOS DE IMPLEMENTACIÓN DE LA MEJORA}
%Introducción
Este capitulo se centra en el análisis detallado de los costos de la implementación de la mejora. A través de cada punto, se examinaran los costos materiales, mano de obra, equipos entre otros gastos relevantes para la ejecución del proyecto. El objetivo es proporcionar una visión completa de los recursos financieros necesarios para llevar a cabo la mejora con éxito.

\subsection{5.1 Costo de materiales}
%Introducción
En esta sección, se detallan los costos asociados con los materiales necesarios para la implementación de la mejora propuesta. 
\definecolor{ChetwodeBlue}{rgb}{0.556,0.662,0.858}
\begin{table}[H]
\centering
\caption{Costos de materiales}
\begin{tblr}{
  cells = {c},
  row{1} = {ChetwodeBlue},
  vline{-} = {1-5}{},
  vline{4-6} = {6}{},
  hline{1-6} = {-}{},
  hline{7} = {4-5}{},
}
\textbf{Ítem} & \textbf{Descripción} & \textbf{~Cantidad} & \textbf{Costo Unitario} & \textbf{Monto Total}\\
1 & Lapicero & 2 & ~S/. 1.00 & ~S/. 2.00\\
2 & Hojas Bond & 50 & ~S/. 0.10 & ~S/. 5.00\\
3 & Lapiz & 1 & ~S/. 0.50 & ~S/. 0.50\\
4 & Borrador & 1 & ~S/. 1.00 & ~S/. 1.00\\
 &  &  & ~Total~ & ~S/. 8.50
\end{tblr}
\label{tab:costoMateriales}
\end{table}
En la tabla \ref{tab:costoMateriales} se detallaron los costos de materiales identificados para la implementación de la mejora propuesta. Se han incluido ítems como lapiceros, hojas bond, lápices y borradores, junto con su respectiva cantidad y costo unitario. El monto total representa la suma de todos los gastos. Estos costos son útiles para la planificación financiera y asegurar la disponibilidad de los recursos necesarios.  

\subsection{5.2 Costo de Recursos Humanos}
%Introducción
Estos costos se definen como el esfuerzo de carácter físico y mental que un trabajador realiza a cambio de dinero. Es el personal que posee toda organización para llevar a cabo sus actividades empresariales. En este apartado se detallan los costos asociados con el trabajo realizado por el personal involucrado en la implementación de la mejora propuesta.

\subsubsection*{Desglose de Cálculos}
\begin{enumerate}
    \item Analista de Sistemas:
    \begin{itemize}
        \item Sueldo promedio: S/ 2200
        \item Horas trabajadas: 48
        \item Costo por hora: S/ 13.75
        \item Costo total: 48 horas * S/ 13.75/hora = S/ 660.00
    \end{itemize}
    
    
    \item Diseñador de Interfaces:
    \begin{itemize}
        \item Sueldo promedio: S/ 2450
        \item Horas trabajadas: 32
        \item Costo por hora: S/ 15.31
        \item Costo total: 32 horas * S/ 15.31/hora = S/ 489.92
    \end{itemize}
    
    
    \item Programador Fullstack:
    \begin{itemize}
        \item Sueldo promedio: S/ 2900
        \item Horas trabajadas: 152
        \item Costo por hora: S/ 18.13
        \item Costo total: 152 horas * S/ 18.13/hora = S/ 2751.76
    \end{itemize}    
    
    \item Tester:
    \begin{itemize}
        \item Sueldo promedio: S/ 1203
        \item Horas trabajadas: 24
        \item Costo por hora: S/ 7.52
        \item Costo total: 24 horas * S/ 7.52/hora = S/ 180.48
    \end{itemize}
    
\end{enumerate}

\begin{table}
\centering
\caption[Costo de Recursos Humanos]{Costo de Recursos Humanos}
\begin{tblr}{
  cells = {c},
  row{1} = {ChetwodeBlue},
  vline{-} = {1-5}{},
  vline{2-6} = {6}{},
  hline{1-6} = {-}{},
  hline{7} = {2-5}{},
}
\textbf{Ítem} & \textbf{Descripción} & \textbf{Horas - Hombre} & \textbf{Costo Hora} & \textbf{Costo Total}\\
1 & Analista de Sistemas & 48 & S/ 13.8 & S/ 660.0\\
2 & Diseñador Interfaces & 32 & S/ 15.3 & S/
  489.9\\
3 & Programador & 152 & S/ 18.1 & S/ 2,755.8\\
4 & Tester & 24 & S/ 7.5 & S/
  180.5\\
 & Total & 256 &  & S/ 4,086.2
\end{tblr}
\label{tab:costoRH}
\end{table}

La tabla \ref{tab:costoRH} los costos de la mano de obra para la implementación de la mejora propuesta. Se han incluido los roles de analistas, diseñadores de interfaces, programadores y testers junto con la cantidad de horas-hombre dedicadas a cada uno y el costo por hora. \\
El monto total representa la suma de los costos individuales de cada trabajador, obteniendo asi la estimación precisa de los gastos asociados con la mano de obra necesaria para el proyecto. Es importante destacar que el programador genera la mayor inversión debido a su papel fundamental en el desarrollo del proyecto.

\subsection{5.3 Costo de máquinas, herramientas y equipos}
%Introducción
La evaluación de costos relacionados con equipos es importante para presupuestar proyectos de mejora de manera precisa. Este análisis implica el costeo del uso adecuado del equipo para la implementación del trabajo.

\begin{table}[H]
\centering
\caption[Costo de equipos]{Costo de equipos}
\begin{tblr}{
  cells = {c},
  column{2} = {3cm}, 
  row{1} = {ChetwodeBlue},
  vline{-} = {1-2}{},
  vline{5-7} = {3}{},
  hline{1-3} = {-}{},
  hline{4} = {5-6}{},
}
\textbf{Ítem} & \textbf{Uso de máquina} & \textbf{Horas Totales / Mes} & \textbf{Horas Totales} & \textbf{Costo / Hora} & \textbf{Costo Total}\\
1 & {Computadora de
\\~Escritorio} & 128 & 512 & S/ 2.00 & S/ 1,024.00\\
 &  &  &  & Total & S/
  1,024.00
\end{tblr}
\label{tab:costoEquipos}
\end{table}

La tabla \ref{tab:costoEquipos} muestra los costos asociados con el uso del único ítem que se uso durante la implementación de la mejora. El costo total representa la suma de los gastos incurridos por el uso del equipo durante el período definido para el proyecto. 


\subsection{5.4 Otros costos de implementación de la Mejora}
%Introducción
%En esta sección se analizan los costos adicionales igual de relevantes para el proyecto, estos valores son importantes para determinar el costo-beneficio. Se incluyo el aspecto economico del uso de energia que no fue abordado anteriormente.
En esta sección, se analizan los costos adicionales que son igual de relevantes para el proyecto. Estos valores son esenciales para determinar la relación costo-beneficio y asegurar la viabilidad económica de la implementación. Se ha incluido una evaluación detallada del costo de la energía eléctrica consumida, la suscripción a la API de SUNAT, y el servicio de Internet, los cuales son factores clave que no se abordaron en secciones anteriores. La identificación y cuantificación de estos costos son cruciales para una evaluación completa y precisa de los recursos necesarios para llevar a cabo la mejora propuesta.

\begin{table}[H]
\centering
\caption[Costo de servicios]{Costo de servicios}
\begin{tblr}{
  cells = {c},
  row{1} = {ChetwodeBlue},
  vline{-} = {1-4}{},
  vline{5-7} = {5}{},
  hline{1-5} = {-}{},
  hline{6} = {5-6}{},
}
\textbf{Ítem} & \textbf{Descripción} & \textbf{Unidad} & \textbf{Precio Unitario} & \textbf{Cantidad} & \textbf{Costo Total}\\
1 & Api Sunat & - & - & 1 & S/ 50.00\\
1 & Internet & - & - & mensual & S/
  90.00\\
1 & Energía eléctrica & 0.22 & S/ 0.50 & 128 & S/ 14.00\\
 &  &  &  & Total & S/
  154.00
\end{tblr}
\label{tab:costoServicios}
\end{table}

La tabla \ref{tab:costoServicios} detalla los costos de servicios adicionales asociados a la implementación del proyecto. Estos costos incluyen la suscripción a servicios externos necesarios para la operación del sistema y el consumo de recursos energéticos. Cada ítem está desglosado en su unidad correspondiente, precio unitario, cantidad requerida y el costo total asociado. Presenta una vista detallada de los costos estimados relacionados con la implementación del proyecto de mejora. Se incluyen los siguientes elementos:

\begin{itemize}
    \item API SUNAT: El costo de la suscripción anual a la API de SUNAT, necesaria para acceder a información tributaria y cumplir con las normativas fiscales.
    \item Internet: El costo mensual del servicio de Internet, indispensable para la conectividad y el funcionamiento de los sistemas en línea.
    \item Energía eléctrica consumida estimada: El costo asociado al consumo de energía eléctrica durante el desarrollo del proyecto, calculado en kilovatios-hora (kWh). Se ha aplicado un costo promedio de S/ 0.50 por kWh para estimar el consumo total de la computadora a lo largo del período del proyecto.

\end{itemize}


%La tabla muestra el costo estimado de la energía eléctrica consumida durante la implementación del proyecto de mejora. Se ha calculado el consumo estimado en kilovatios-hora(kWh) y se ha aplicado un costo por hora, considerando el valor de kilovatio-hora en Arequipa. En este caso, se estimo el consumo total de energía de la computador durante el tiempo definido del proyecto.

\subsection{5.5 Costo total de la implementación de la Mejora}
%Introducción
Esta sección presenta el resumen de todos los costos propuestos para la implementación de la mejora. Es costo total es un indicativo del valor estimado necesario para ejecutar las mejoras. Conocer estos costos ayudara a comprender la viabilidad económica del proyecto y asi determinar la factibildidad.

\begin{table}[H]
\centering
\caption{Coste Total}
\begin{tblr}{
  cells = {c},
  row{1} = {ChetwodeBlue},
  vline{-} = {1-5}{},
  vline{2-4} = {6}{},
  hline{1-6} = {-}{},
  hline{7} = {2-3}{},
}
\textbf{N°} & \textbf{Descripción} & \textbf{Costo Total}\\
1 & Costo materiales & S/ 8.50\\
2 & Costo de RRHH & S/
  4,086.16\\
3 & Costo de equipos & S/ 1,024.00\\
4 & Coste de servicios & S/
  154.00\\
 & Total & S/ 5,272.66
\end{tblr}
\end{table}

La tabla resume los costos totales considerando los costos de materiales, mano de obra, equipos y el costo de energía eléctrica como componentes principales del presupuesto total, de esta forma se proporciona una vista general del gasto previsto para la mejora.

