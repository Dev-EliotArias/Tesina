\newpage
\section{CAPITULO VI}
\section*{EVALUACIÓN TÉCNICA Y ECONÓMICA DE LA MEJORA}
En este capitulo se evaluará técnica y económicamente la propuesta de mejora para determinar su factibilidad de aplicación. Para ello, se utilizará el método de análisis Costo/Beneficio, el cual es el proceso de monetizar los diferentes costos y beneficios de una actividad. Este análisis permite estimar el impacto financiero acumulado de lo que se desea lograr y evaluar si el proyecto es rentable o no.
La relación costo-beneficio (B/C), también conocida como índice neto de rentabilidad, se obtiene dividiendo el Valor Actual de los Ingresos totales netos o beneficios netos (VAI) entre el Valor Actual de los Costos de inversión o costos totales (VAC) de un proyecto:

\[\frac{Costo}{Beneficio} = \frac{VAC}{VAI}\]

Según el análisis Costo/Beneficio, un proyecto será rentable cuando la relación es mayor que 1. Si es igual o menor que 1, el proyecto no es viable, ya que los beneficios serán iguales o menores que los costos de inversión.

\subsection{6.1 Beneficio técnico y/o económico esperado de la Mejora}
En este punto se evaluará el beneficio técnico y económico esperado de la mejora propuesta.
Actualmente, se considera que el tiempo invertido en registrar a un nuevo cliente junto con su automóvil de forma escrita en la empresa representa una pérdida, ya que este tiempo podría utilizarse de manera más productiva. Lo mismo ocurre con los mecánicos y encargados del servicio, quienes a veces tardan más de lo habitual en realizar sus tareas debido a la falta de un control preciso del tiempo para cada una de ellas.\\

Uno de los beneficios más significativos se encuentra en el ahorro de horas-hombre por día. A diferencia del registro escrito y la asignación oral de tareas, el sistema se encargará de registrar y almacenar automáticamente la información de los nuevos clientes y vehículos, así como de verificar si el cliente ya está registrado. Además, gestionará la asignación automática de tareas para los técnicos disponibles en el taller.\\


\begin{table}[H]
\caption{Sistema Actual}
\centering
\begin{tblr}{
  cells = {c},
  column{1} = {2.5cm},
  column{2} = {2cm},
  column{3} = {2.2cm},
  column{4} = {2.5cm},  
  column{5} = {2.8cm},
  row{1} = {ChetwodeBlue},
  vline{-} = {1-2}{},
  vline{4-7} = {3}{},
  hline{1-3} = {-}{},
  hline{4} = {4-6}{},
}
\textbf{Personal} & \textbf{Atenciones / día} & \textbf{Promedio duración} & \textbf{Total / dia} & \textbf{Total / semana} & \textbf{Total Mes}\\
Administrador & 15 & 32 min & 480 min & 2880 min & 11520 min\\
 &  &  & 8 h & 48 h & 192 h
\end{tblr}
\end{table}
Con el sistema actual, donde se realiza un promedio de 15 atenciones al día, el tiempo empleado en cada atención ese de 112 minutos, pero sin considerar el tiempo del servicio e inspección se reduce a 32 minutos que son lo único que se considerara para el servicio al cliente.



 \begin{table}[H]
\caption[Sistema Mejorado]{Sistema Mejorado}
\centering
\begin{tblr}{
  row{1} = {ChetwodeBlue},
  row{2} = {l},
  cells = {c},
  column{1} = {2.5cm},
  column{2} = {2cm},
  column{3} = {2.2cm},
  column{4} = {2.5cm},  
  column{5} = {2.8cm},
  vline{-} = {1-2}{},
  vline{4-7} = {3}{},
  hline{1-3} = {-}{},
  hline{4} = {4-6}{},
}
\textbf{Personal} & \textbf{Atenciones / día} & \textbf{Promedio duración} & \textbf{Total  / dia} & \textbf{Total / semana} & \textbf{Total Mes}\\
Administrador & 15 & 14 min & 210 min & 1260 min & 5040 min\\
& & & 3.5 h & 21 h & 84 h\\
\end{tblr}
\end{table}

Con el sistema mejorado, el tiempo empleado en cada atención fue reducido de 32 minutos a 14, ahorrando un total de 18 minutos por cada atención realizada, esto claro sin contar la parte del brindado del servicio e inspección.


\begin{table}[H]
\caption[Resumen]{Resumen}
\centering
\begin{tblr}{
  row{1} = {ChetwodeBlue},
  cells = {c},
  hlines,
  vlines,
}
\textbf{Sistema Actual} & \textbf{Sistema Mejorado} & \textbf{Ahorro en Tiempo/Administrador}\\
11520 min & 5040 min & 6480 min\\
192 h & 84 h & 108 horas
\end{tblr}
\end{table}

En resumen el total durante el mes es de 6840 minutos o 108 horas que corresponde a el tiempo ahorrado en la atención, registro y consulta implementado por la mejora.


\newpage
\subsection{6.2 Relación Beneficio/Costo}
A continuación se presentara el costo beneficio de aplicar la mejora, para dicho cálculo se tomara 
como base los salarios del administrador. Se analizará la relación costo-beneficio de la mejora que se propuso, considerando tanto los costos iniciales como los costos de mantenimiento a lo largo del tiempo, así como los beneficios económicos derivados del ahorro en tiempo / recursos humanos. Este análisis nos permite evaluar la viabilidad económica de la implementación de la mejora en diferentes periodos de tiempo.
\begin{itemize}
    \item Pago mensual al administrador: S/ 2416
    \item Pago por hora al administrador: 
    \[\frac{\text{S/} 2416}{192 \text{ horas}} = \text{S/} 12.58 \text{ por hora}\]
\end{itemize}
\begin{table}
\centering
\caption{Ahorro total con el metodo mejorado}
\begin{tblr}{
  cells = {c},
  row{1} = {ChetwodeBlue},
  hlines,
  vlines,
}
\textbf{Descripción} & \textbf{Horas Hombre S/.} & \textbf{Horas Trabajadas/Mes} & \textbf{Monto Ahorrado}\\
Administrador & 12.58 & 108 & 1358.6
\end{tblr}
\end{table}


La Tabla 20 presenta los costos asociados a la implementación del proyecto de mejora. Se consideran diferentes categorías de costos, como materiales, recursos humanos (RRHH), equipos y servicios.

\begin{itemize}
    \item Materiales: Gastos iniciales en materiales necesarios para la implementación.
    \item RRHH: Costos mensuales del personal administrativo, calculados en función del tiempo ahorrado y su costo por hora.
    \item Equipos: Costos mensuales asociados con el uso de equipos necesarios para la mejora.
    \item Servicios: Costos mensuales de servicios relacionados con la implementación de la mejora.
\end{itemize}

Estos costos se calculan y se presentan para cada mes durante el período de desarrollo, mostrando tanto los costos mensuales como los costos acumulados.

\begin{table}[H]
\centering
\caption{Costo}
\begin{tblr}{
  cells = {c},
  row{1} = {ChetwodeBlue},
  row{4} = {red},
  row{9} = {fg=red},
  row{10} = {fg=red},
  row{11} = {ChetwodeBlue},
  cell{1}{1} = {c=6}{},
  cell{2}{2} = {c=5}{},
  cell{4}{1} = {c=6}{},
  cell{5}{1} = {fg=red},
  cell{5}{2} = {fg=red},
  cell{6}{1} = {fg=red},
  cell{6}{3} = {fg=red},
  cell{6}{4} = {fg=red},
  cell{6}{5} = {fg=red},
  cell{6}{6} = {fg=red},
  cell{7}{1} = {fg=red},
  cell{7}{3} = {fg=red},
  cell{7}{4} = {fg=red},
  cell{7}{5} = {fg=red},
  cell{7}{6} = {fg=red},
  cell{8}{1} = {fg=red},
  cell{8}{3} = {fg=red},
  cell{8}{4} = {fg=red},
  cell{8}{5} = {fg=red},
  cell{8}{6} = {fg=red},
  cell{11}{1} = {c=6}{},
  cell{12}{2} = {fg=blue},
  cell{12}{3} = {fg=blue},
  cell{12}{4} = {fg=blue},
  cell{12}{5} = {fg=blue},
  cell{12}{6} = {fg=blue},
  cell{13}{2} = {fg=blue},
  cell{13}{3} = {fg=blue},
  cell{13}{4} = {fg=blue},
  cell{13}{5} = {fg=blue},
  cell{13}{6} = {fg=blue},
  cell{14}{2} = {fg=blue},
  cell{14}{3} = {fg=blue},
  cell{14}{4} = {fg=blue},
  cell{14}{5} = {fg=blue},
  cell{14}{6} = {fg=blue},
  cell{16}{2} = {fg=red},
  cell{16}{3} = {fg=red},
  cell{16}{4} = {fg=red},
  cell{16}{5} = {fg=red},
  cell{16}{6} = {fg=red},
  cell{17}{2} = {fg=red},
  cell{17}{3} = {fg=red},
  cell{17}{4} = {fg=red},
  cell{17}{5} = {fg=red},
  cell{17}{6} = {fg=red},
  hlines,
  vlines,
}
Cuadro Costos Beneficio &  &  &  &  & \\
~ & Periodo Desarrollo &  &  &  & \\
Periodo (Mes) & ~0 & ~1 & ~2 & ~3 & ~4\\
\textbf{Costos}&  &  &  &  & \\
Materiales & S/ 8.50 & \textcolor{red}{~} & \textcolor{red}{~} & \textcolor{red}{~} & \textcolor{red}{~}\\
RRHH & \textcolor{red}{~} & S/  1,021.54 & S/  1,021.54 & S/  1,021.54 & S/  1,021.54\\
Equipos & \textcolor{red}{~} & S/ 128.00 & S/ 128.00 & S/ 128.00 & S/ 128.00\\
Servicios~ & \textcolor{red}{~} & S/ 154.00 & S/ 154.00 & S/ 154.00 & S/ 154.00\\
\textbf{Total} & S/ 8.50 & S/  1,303.54 & S/  1,303.54 & S/  1,303.54 & S/  1,303.54\\
\textbf{Acumulado} & S/ 8.50 & S/  1,312.04 & S/  2,615.58 & S/  3,919.12 & S/  5,222.66\\
\textbf{Beneficios} &  &  &  &  & \\
Beneficios Indirectos & S/ 0.00 & S/ 0.00 & S/ 0.00 & S/ 0.00 & S/ 0.00\\
Total & S/ 0.00 & S/ 0.00 & S/ 0.00 & S/ 0.00 & S/ 0.00\\
Acumulado & S/ 0.00 & S/ 0.00 & S/ 0.00 & S/ 0.00 & S/ 0.00\\
 &  &  &  &  & \\
\textbf{Total} & -S/ 8.50 & -S/ 1,303.54 & -S/ 1,303.54 & -S/ 1,303.54 & -S/ 1,303.54\\
\textbf{Acumulado} & -S/ 8.50 & -S/  1,312.04 & -S/  2,615.58 & -S/  3,919.12 & -S/  5,222.66
\end{tblr}
\end{table}

%El costo total de la implementación de la mejora hasta el final del periodo de desarrollo fue de S\/5222.66

La Tabla 21 muestra los beneficios económicos derivados del ahorro en tiempo/hombre. Se enfoca principalmente en los beneficios indirectos, que en este caso se derivan del tiempo ahorrado por el administrador.


\begin{itemize}
    \item Beneficios Indirectos: Ahorro en costos laborales debido a la reducción de tiempo necesario para registrar clientes y realizar otras tareas administrativas.
\end{itemize}

Los beneficios se calculan mensualmente y se presentan acumulados para demostrar cómo el ahorro se incrementa con el tiempo, compensando los costos de implementación iniciales.

\begin{table}[H]
\centering
\caption{Beneficio}
\begin{tblr}{
  cells = {c},
  row{1} = {ChetwodeBlue},
  row{4} = {red},
  row{7} = {fg=red},
  row{8} = {fg=red},
  row{9} = {fg=red},
  row{10} = {fg=red},
  row{11} = {ChetwodeBlue},
  cell{1}{1} = {c=6}{},
  cell{2}{1} = {c=6}{},
  cell{4}{1} = {c=6}{},
  cell{5}{1} = {fg=red},
  cell{6}{1} = {fg=red},
  cell{11}{1} = {c=6}{},
  cell{12}{2} = {fg=blue},
  cell{12}{3} = {fg=blue},
  cell{12}{4} = {fg=blue},
  cell{12}{5} = {fg=blue},
  cell{12}{6} = {fg=blue},
  cell{13}{2} = {fg=blue},
  cell{13}{3} = {fg=blue},
  cell{13}{4} = {fg=blue},
  cell{13}{5} = {fg=blue},
  cell{13}{6} = {fg=blue},
  cell{14}{2} = {fg=blue},
  cell{14}{3} = {fg=blue},
  cell{14}{4} = {fg=blue},
  cell{14}{5} = {fg=blue},
  cell{14}{6} = {fg=blue},
  cell{16}{2} = {fg=blue},
  cell{16}{3} = {fg=blue},
  cell{16}{4} = {fg=blue},
  cell{16}{5} = {fg=blue},
  cell{16}{6} = {fg=blue},
  cell{17}{2} = {fg=red},
  cell{17}{3} = {fg=red},
  cell{17}{4} = {fg=red},
  cell{17}{5} = {fg=red},
  cell{17}{6} = {green,fg=blue},
  hlines,
  vlines,
}
\textbf{Costos Beneficio } &  &  &  &  & \\
\textbf{Periodo Producción } &  &  &  &  & \\
Periodo (Mes) & 5 & 6 & 7 & 8 & 9\\
\textbf{Costos}~ &  &  &  &  & \\
Materiales & \textcolor{red}{~} & \textcolor{red}{~} & \textcolor{red}{~} & \textcolor{red}{~} & \textcolor{red}{~}\\
RRHH & \textcolor{red}{~} & \textcolor{red}{~} & \textcolor{red}{~} & \textcolor{red}{~} & \textcolor{red}{~}\\
Equipos & S/ 20.00 & S/ 20.00 & S/ 20.00 & S/ 20.00 & S/ 20.00\\
Servicios~ & S/ 154.00 & S/ 154.00 & S/ 154.00 & S/ 154.00 & S/ 154.00\\
Total & S/ 174.00 & S/ 174.00 & S/ 174.00 & S/ 174.00 & S/ 174.00\\
Acumula & S/  5,396.66 & S/  5,570.66 & S/  5,744.66 & S/  5,918.66 & S/  6,092.66\\
\textbf{Beneficios} &  &  &  &  & \\
Beneficios Indirectos & S/  1,358.00 & S/  1,358.00 & S/  1,358.00 & S/  1,358.00 & S/  1,358.00\\
Total & S/  1,358.00 & S/  1,358.00 & S/  1,358.00 & S/  1,358.00 & S/  1,358.00\\
Acumulado & S/  1,358.00 & S/  2,716.00 & S/  4,074.00 & S/  5,432.00 & S/  6,790.00\\
 &  &  &  &  & \\
Total & S/ 1,184.00 & S/ 1,184.00 & S/ 1,184.00 & S/ 1,184.00 & S/ 1,184.00\\
\textbf{Acumulado} & -S/  4,038.66 & -S/  2,854.66 & -S/  1,670.66 & -S/  486.66 & S/ 697.34
\end{tblr}
\end{table}


\subsubsection*{Costo - Beneficio para el primer mes:}
\begin{itemize}
    \item Costo Acumulado en el Primer Mes: S/ 5396.66 (S/ 5222.66 de costos acumulados + S/ 174.00 de servicios adicionales en el primer mes)
    \item Beneficio en el Primer Mes: S/ 1358.6
\end{itemize}
\[\frac{B}{C} = \frac{Beneficios}{Costos} = \frac{S/ 5,396.66}{S/ 1,358.00} = 0.25\]


\subsubsection*{Costo-Beneficio en el Primer Trimestre:}
\begin{itemize}
    \item Costo Acumulado en el Primer Trimestre: S/ 5,744.66 (Costos acumulados)
    \item Beneficio en el Primer Trimestre: S/ 4,074.00 (S/ 1358.6 por mes x 3 meses)
\end{itemize}
\[\frac{B}{C} = \frac{Beneficios}{Costos} = \frac{S/ 4,074.00}{S/ 5,744.66} = 0.70\]

\subsubsection*{Costo-Beneficio en el quinto mes:}
\begin{itemize}
    \item Costo Acumulado: S/ 6,092.66 (Costos acumulados)
    \item Beneficio en el Primer Trimestre: S/ 6,790.00 (S/ 1358.6 por mes x 5 meses)
\end{itemize}
\[\frac{B}{C} = \frac{Beneficios}{Costos} = \frac{S/ 6,790.00}{S/ 6,092.66} = 1.11\]

\paragraph*{En conclución} el proyecto es viable a partir del quinto mes despues de la implementación de la mejora.
